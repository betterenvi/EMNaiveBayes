%-*- coding: UTF-8 -*-
\documentclass[UTF8]{ctexart}
% 使得打字机粗体可以被使\textbf{}用。
\usepackage{lmodern}
\usepackage{amsmath}
% 产生 originauth.tex 里的 \square。
\usepackage{amssymb}
\DeclareMathOperator*{\argmax}{argmax}
% 提供 Verbatim 环境和 \VerbatimInput 命令。
\usepackage{fancyvrb}
\title{机器学习第三次作业报告}
\author{XXX}
\date{\today}

\bibliographystyle{plain}

\begin{document}
\maketitle

推导

假定共有$M$个特征,第$j$个特征有$q_j$个可能的取值,共有$K$个类。对于单个样本数据$\boldsymbol{x}=(x^{(1)}, x^{(2)}, \cdots, x^{(M)})^T$,由贝叶斯公式有$$
\hat{y} = f(\boldsymbol{x}) = \mathop{\argmax}_y {P(y|\boldsymbol{x})} = \mathop{\argmax}_y {P(\boldsymbol{x}|y) P(y)},
$$
再由条件独立性假设,有$$
\hat{y} = f(\boldsymbol{x}) = \mathop{\argmax}_y {P(y) \prod\limits_{j=1}^{M} P(x^{(j)}|y)}
$$

现有观测数据$\boldsymbol{X} = (\boldsymbol{x_1}, \boldsymbol{x_2}, \cdots, \boldsymbol{x_N})^T$,$N$是样本个数。完全数据是$$
(u_{i}^{(1,1)}, u_{i}^{(1,2)}, \cdots, u_{i}^{(1,q_1)}, u_{i}^{(2,1)}, u_{i}^{(2,2)}, \cdots, u_{i}^{(2,q_2)}, \cdots, u_{i}^{(M,1)}, u_{i}^{(M,2)}, \cdots, u_{i}^{(M,q_M)}, \\ v_{i1}, v_{i2}, \cdots, v_{iK}), i = 1, 2, \cdots, N
$$

\begin{align}
\label{def:nCommit}
nCommit(prj) &= \vert \{cmt \in allCommits(prj): T_1 \leq commitAt(cmt, prj) \textless T_3\} \vert
\end{align}

%\section{X}
%\section{但是}

\end{document}
